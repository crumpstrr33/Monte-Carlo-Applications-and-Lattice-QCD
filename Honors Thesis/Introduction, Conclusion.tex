\documentclass[11pt]{article}
\usepackage[top=1in, bottom=1in, left=1in, right=1in]{geometry}
\usepackage{amssymb}
\usepackage{physics}

\usepackage{fancyhdr}
\pagestyle{fancy}
\renewcommand{\sectionmark}[1]{\markboth{#1}{}}
\fancyhf{}
\fancyhead[R]{\leftmark}
\fancyhead[L]{Monte Carlo and Lattice QCD}
\fancyfoot[C]{\thepage}
\fancypagestyle{plain}{\fancyhf{}
	\renewcommand{\headrulewidth}{0pt}}

\begin{document}
\section{Introduction}
The myriad methods developed in the sciences and mathematics, no matter how esoteric, come about due to a problem that is unsolved that scientists or mathematicians want to solve. Problems arise due to new information, new measurements or even just a new idea that one person may have and these problems are solved using one or a number of these methods or new methods are discovered.

The field of lattice quantum chromodynamics (QCD) is a good example of this. Lattice QCD is the study of the strong force through the use of discretized spacetime lattice simulations and is considered a nonperturbative theory. Because of this, more traditional methods can not be used. In electrodynamics, the coupling constant (the constant that determines the strength of the force) is the fine-structure constant:
\begin{align}
	\alpha=\frac{1}{4\pi\epsilon_0}\frac{e^2}{\hbar c}\approx\frac{1}{137}.
\end{align}
Because of its extremely small value, perturbative methods can be implemented and approximations can be made that are in the form of series of powers of $\alpha$. Such series would be dominated by the first couple terms and even first-order solutions can be very accurate. Whereas in QCD, the coupling constant $g$ is of the order of one and therefore perturbation theory is not a valid method for low energy systems but lattice QCD is. Lattice QCD is not perturbative but rather finds its power in the form of computers.

Lattice QCD is a discrete and numerical approximation of the continuum and as such, when the spacing between lattice sites $a\to0$, we return to the continuous theory. Because of this, the error that arises comes from this discretization and also from the numerical methods used in this paper. A common compromise is the choice for lattice size. Too large and the program can take too long to run to be practical. Too small and the error on the data is large. 

The numerical methods that are used in this paper are Monte Carlo methods. These methods rely on random sampling in the hopes that doing so will eventually approximate the equilibrium state of a system. The method used specifically in the evolution of the systems is the Metropolis algorithm, an algorithm that decides whether a change in the system should be accepted or rejected. This decision is based on the physics of the system. For example, if a system tends towards the lowest energy state, then the Metropolis algorithm is used to accept states that are a lower energy. Monte Carlo methods are also used to approximate multi-dimensional integrals to measure specific observables of a system. As such, this paper highlights the flexibility and power of the aforementioned methods and shows how measurements are relatively simple once a system is properly modeled.

In this paper, I will explore simpler systems using the methods I will then use for lattice QCD. These systems are the Ising model and the harmonic oscillator. In every simulation, the same methods are used with the only significant changes coming in the form of the physics of the specific systems. Even though these systems are vastly different, the methods are the same.

\section{Conclusion and Remarks}
In this paper, we explored the use of Monte Carlo techniques on physical models. I used the Python language because of my previous experiences with it and its convenience especially with mathematical operations, multidimensional arrays and useful libraries such as \texttt{numpy} and \texttt{matplotlib}. Going forward, implementing these models in a language such as C would be an advantageous endeavor since the flaw of Python is its inefficiency especially compared to C and C-like languages. Also, the use of a Message Passing Interface (MPI) and/or multithreading would also help with the time inconveniences faced with lattice QCD models. And with the lack of time, more significant measurements and more accurate lattice QCD models were not able to be explored as was hoped. As such, this may be a future project to expand on what I have learned thus far in lattice QCD.

That being said, we have seen the flexibility and power that some Monte Carlo methods have. The use of the Metropolis algorithm, assuming ergodicity and detailed balance, is extremely powerful because of its ability to converge on a desired equilibrium. And, due to taking a large sample size, possibly complex expectation values and measurements can be accurately approximated if simplified to a mean. As is shown in \cite{MainPaper}, for some functional $\Gamma[x]$ such that
\begin{align}
	\langle\Gamma[x]\rangle=\frac{\int\mathcal{D}[x(t)]\Gamma[x]e^{-S[x]}}{\int\mathcal{D}[x(t)]e^{-S[x]}},
\end{align}
we can instead approximate it by
\begin{align}
	\langle\Gamma[x]\rangle\approx\overline{\Gamma}\equiv\frac{1}{N}\sum_{i=1}^N\Gamma[x^{(\alpha)}]
\end{align}
if the probability for some path $x^{(\alpha)}$ is equal to the weight of the expectation value (in this case $e^{-S[x]}$. The above can be very easily computed in a program and the error comes just from the sample size $N$.

Also, this paper has yielded some satisfying results. The expectation of the magnetization and susceptibility in the Ising model was modeled successfully, as with the excitation energy $E_1-E_0$ of the harmonic oscillator. The latter can be solved by the Schr\"odinger equation but it was satisfying to see the same result through the use of Feynman path integrals and the Metropolis algorithm, a completely different approach. Lastly, three different Wilson loops, $a\cross a$, $a\cross2a$ and $a\cross3a$, were calculated using an improved Wilson action. Nevertheless, such a measurement is just the beginning; measurements such as a static quark, anti-quark potential and many others could next be calculated.
\end{document}